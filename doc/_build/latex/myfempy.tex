%% Generated by Sphinx.
\def\sphinxdocclass{report}
\documentclass[letterpaper,10pt,english]{sphinxmanual}
\ifdefined\pdfpxdimen
   \let\sphinxpxdimen\pdfpxdimen\else\newdimen\sphinxpxdimen
\fi \sphinxpxdimen=.75bp\relax
\ifdefined\pdfimageresolution
    \pdfimageresolution= \numexpr \dimexpr1in\relax/\sphinxpxdimen\relax
\fi
%% let collapsible pdf bookmarks panel have high depth per default
\PassOptionsToPackage{bookmarksdepth=5}{hyperref}

\PassOptionsToPackage{warn}{textcomp}
\usepackage[utf8]{inputenc}
\ifdefined\DeclareUnicodeCharacter
% support both utf8 and utf8x syntaxes
  \ifdefined\DeclareUnicodeCharacterAsOptional
    \def\sphinxDUC#1{\DeclareUnicodeCharacter{"#1}}
  \else
    \let\sphinxDUC\DeclareUnicodeCharacter
  \fi
  \sphinxDUC{00A0}{\nobreakspace}
  \sphinxDUC{2500}{\sphinxunichar{2500}}
  \sphinxDUC{2502}{\sphinxunichar{2502}}
  \sphinxDUC{2514}{\sphinxunichar{2514}}
  \sphinxDUC{251C}{\sphinxunichar{251C}}
  \sphinxDUC{2572}{\textbackslash}
\fi
\usepackage{cmap}
\usepackage[T1]{fontenc}
\usepackage{amsmath,amssymb,amstext}
\usepackage{babel}



\usepackage{tgtermes}
\usepackage{tgheros}
\renewcommand{\ttdefault}{txtt}



\usepackage[Bjarne]{fncychap}
\usepackage{sphinx}

\fvset{fontsize=auto}
\usepackage{geometry}


% Include hyperref last.
\usepackage{hyperref}
% Fix anchor placement for figures with captions.
\usepackage{hypcap}% it must be loaded after hyperref.
% Set up styles of URL: it should be placed after hyperref.
\urlstyle{same}


\usepackage{sphinxmessages}
\setcounter{tocdepth}{2}



\title{myfempy}
\date{Jul 27, 2022}
\release{}
\author{Antonio Vinicius Garcia Campos.\@{}}
\newcommand{\sphinxlogo}{\vbox{}}
\renewcommand{\releasename}{}
\makeindex
\begin{document}

\ifdefined\shorthandoff
  \ifnum\catcode`\=\string=\active\shorthandoff{=}\fi
  \ifnum\catcode`\"=\active\shorthandoff{"}\fi
\fi

\pagestyle{empty}
\sphinxmaketitle
\pagestyle{plain}
\sphinxtableofcontents
\pagestyle{normal}
\phantomsection\label{\detokenize{index::doc}}
\sphinxAtStartPar
\sphinxstylestrong{IN DEVELOPMENT !}

\begin{figure}[htbp]
\centering

\noindent\sphinxincludegraphics{{logoB}.png}
\end{figure}



\sphinxAtStartPar
Copyright © Antonio Vinicius G. Campos and 3D EasyCAE, 2022


\chapter{About}
\label{\detokenize{index:about}}
\sphinxAtStartPar
\sphinxstylestrong{Myfempy} is a python package based on finite element method for
scientific analysis. The code is open source and \sphinxstyleemphasis{intended for
educational and scientific purposes only, not recommended to commercial
use}. You can help us by contributing with a donation on the main
project page, read the support options. \sphinxstylestrong{If you use myfempy in your
research, the developers would be grateful if you could cite in your
work.}


\chapter{Installation}
\label{\detokenize{index:installation}}

\section{To install myfempy manually in your directory, following the steps}
\label{\detokenize{index:to-install-myfempy-manually-in-your-directory-following-the-steps}}\begin{enumerate}
\sphinxsetlistlabels{\arabic}{enumi}{enumii}{}{.}%
\item {} 
\sphinxAtStartPar
Download the main code from
\sphinxhref{https://github.com/easycae-3d/myfempy/tree/main}{github/myfempy/main}

\item {} 
\sphinxAtStartPar
Unzip the pack in your preferred location

\item {} 
\sphinxAtStartPar
In the \sphinxstylestrong{myfempy\sphinxhyphen{}main} folder, open a terminal and enter with the
command:

\end{enumerate}

\begin{sphinxVerbatim}[commandchars=\\\{\}]
pip install .
\end{sphinxVerbatim}

\sphinxAtStartPar
\sphinxstyleemphasis{Note: is recommend to create a new virtual environment previously the
installation of}\sphinxstylestrong{myfempy}\sphinxstyleemphasis{and dependencies packs. You can use
the}\sphinxhref{https://virtualenv.pypa.io/en/latest/}{virtualenv}


\chapter{Dependencies}
\label{\detokenize{index:dependencies}}
\sphinxAtStartPar
\sphinxstylestrong{Myfempy} can be used in systems based on Linux, MacOS and Windows.
\sphinxstylestrong{Myfempy} requires Python 3.


\section{Installation prerequisites, required to build \sphinxstylestrong{myfempy}:}
\label{\detokenize{index:installation-prerequisites-required-to-build-myfempy}}\begin{itemize}
\item {} 
\sphinxAtStartPar
\sphinxhref{https://www.python.org/}{Python 3.x} \sphinxhyphen{} \sphinxstyleemphasis{Python is a programming
language that lets you work quickly and integrate systems more
effectively.}

\item {} 
\sphinxAtStartPar
\sphinxhref{https://www.anaconda.com/}{Anaconda} \sphinxhyphen{} \sphinxstyleemphasis{Anaconda offers the
easiest way to perform Python/R data science and machine learning on
a single machine.}

\end{itemize}


\subsection{Outhers prerequisites}
\label{\detokenize{index:outhers-prerequisites}}\begin{itemize}
\item {} 
\sphinxAtStartPar
\sphinxhref{https://gmsh.info/}{Gmsh} \sphinxhyphen{} Gmsh is an open source 3D finite
element mesh generator with a built\sphinxhyphen{}in CAD engine and post\sphinxhyphen{}processor.
\sphinxstyleemphasis{Note: needed install manually}

\end{itemize}


\section{Python packages required for using \sphinxstylestrong{myfempy}:}
\label{\detokenize{index:python-packages-required-for-using-myfempy}}\begin{itemize}
\item {} 
\sphinxAtStartPar
\sphinxhref{https://numpy.org/}{numpy} \sphinxhyphen{} The fundamental package for
scientific computing with Python

\item {} 
\sphinxAtStartPar
\sphinxhref{https://scipy.org/}{scipy} \sphinxhyphen{} Fundamental algorithms for
scientific computing in Python

\item {} 
\sphinxAtStartPar
\sphinxhref{https://vedo.embl.es/}{vedo} \sphinxhyphen{} A python module for scientific
analysis and visualization of эd objects

\end{itemize}

\begin{sphinxVerbatim}[commandchars=\\\{\}]
pip install numpy, scipy, vedo
\end{sphinxVerbatim}


\chapter{Documentation}
\label{\detokenize{index:documentation}}
\sphinxAtStartPar
The project is documented using Sphinx under \sphinxcode{\sphinxupquote{docs/}}. Built version
can be found from \sphinxhref{https://myfempy.readthedocs.io/}{Read the Docs}.
Here are direct links to additional resources:

\sphinxAtStartPar
The \sphinxstylestrong{GitHub/Download} page is available
\sphinxhref{https://github.com/easycae-3d/myfempy/}{here}.

\sphinxAtStartPar
The \sphinxstylestrong{User’s Manual {[}PT\sphinxhyphen{}BR{]}} is available
\sphinxhref{https://github.com/easycae-3d/myfempy/blob/master/docs/Users\_Manual\_PT-BR.pdf}{here}.

\sphinxAtStartPar
Many \sphinxstylestrong{examples} are available
\sphinxhref{https://github.com/easycae-3d/myfempy/tree/master/examples}{here}.


\chapter{Release}
\label{\detokenize{index:release}}
\sphinxAtStartPar
The version up to date is available
\sphinxhref{https://github.com/easycae-3d/myfempy/releases}{here}

\sphinxAtStartPar
Go to \sphinxstyleemphasis{Features List/Version History} to visualization all versions
releses.


\chapter{Features}
\label{\detokenize{index:features}}
\sphinxAtStartPar
\sphinxhref{https://docs.google.com/spreadsheets/d/1k9kiXk2PPuUvcsiukAni005zQc-IOCmP2r-Z6B02304/edit?usp=sharing}{Features
List}


\chapter{License}
\label{\detokenize{index:license}}
\sphinxAtStartPar
\sphinxstylestrong{myfempy} is published under the \sphinxhref{https://en.wikipedia.org/wiki/GNU\_General\_Public\_License}{GPLv3
license}




\chapter{Citing}
\label{\detokenize{index:citing}}
\sphinxAtStartPar
Have you found this software useful for your research? Star the project
and cite it as:
\begin{itemize}
\item {} 
\sphinxAtStartPar
APA:

\begin{sphinxVerbatim}[commandchars=\\\{\}]
\PYG{n}{Antonio} \PYG{n}{Vinicius} \PYG{n}{Garcia} \PYG{n}{Campos}\PYG{o}{.} \PYG{p}{(}\PYG{l+m+mi}{2022}\PYG{p}{)}\PYG{o}{.} \PYG{n}{easycae}\PYG{o}{\PYGZhy{}}\PYG{l+m+mi}{3}\PYG{n}{d}\PYG{o}{/}\PYG{n}{myfempy}\PYG{p}{:} \PYG{n}{beta} \PYG{p}{(}\PYG{n}{v1}\PYG{l+m+mf}{.0}\PYG{l+m+mf}{.1}\PYG{p}{)}\PYG{o}{.} \PYG{n}{Zenodo}\PYG{o}{.} \PYG{n}{https}\PYG{p}{:}\PYG{o}{/}\PYG{o}{/}\PYG{n}{doi}\PYG{o}{.}\PYG{n}{org}\PYG{o}{/}\PYG{l+m+mf}{10.5281}\PYG{o}{/}\PYG{n}{zenodo}\PYG{l+m+mf}{.6376522}
\end{sphinxVerbatim}

\item {} 
\sphinxAtStartPar
BibTex:

\begin{sphinxVerbatim}[commandchars=\\\{\}]
@software\PYG{o}{\PYGZob{}}Campos\PYGZus{}easycae\PYGZhy{}3d\PYGZus{}myfempy\PYGZus{}beta\PYGZus{}2022,
         \PYG{n+nv}{author} \PYG{o}{=} \PYG{o}{\PYGZob{}}Antonio Vinicius Garcia Campos\PYG{o}{\PYGZcb{}},
         \PYG{n+nv}{title} \PYG{o}{=} \PYG{o}{\PYGZob{}}easycae\PYGZhy{}3d/myfempy: beta\PYG{o}{\PYGZcb{}},
         \PYG{n+nv}{version} \PYG{o}{=} \PYG{o}{\PYGZob{}}v1.0.1\PYG{o}{\PYGZcb{}},
         \PYG{n+nv}{url} \PYG{o}{=} \PYG{o}{\PYGZob{}}https://github.com/easycae\PYGZhy{}3d/myfempy/\PYG{o}{\PYGZcb{}},
         \PYG{n+nv}{doi} \PYG{o}{=} \PYG{o}{\PYGZob{}}\PYG{l+m}{10}.5281/zenodo.6376522\PYG{o}{\PYGZcb{}},
         \PYG{n+nv}{month} \PYG{o}{=} \PYG{o}{\PYGZob{}}\PYG{l+m}{3}\PYG{o}{\PYGZcb{}},
         \PYG{n+nv}{year} \PYG{o}{=} \PYG{o}{\PYGZob{}}\PYG{l+m}{2022}\PYG{o}{\PYGZcb{}}
         \PYG{o}{\PYGZcb{}}
\end{sphinxVerbatim}

\end{itemize}


\chapter{References}
\label{\detokenize{index:references}}\begin{itemize}
\item {} 
\sphinxAtStartPar
\sphinxhref{https://myfempy.readthedocs.io/}{Myfempy} \sphinxhyphen{} \sphinxstyleemphasis{A python package for
scientific analysis based on finite element method.}

\item {} 
\sphinxAtStartPar
\sphinxhref{https://en.wikipedia.org/wiki/Finite\_element\_method}{FEM} \sphinxhyphen{} \sphinxstyleemphasis{The
finite element method (FEM) is a popular method for numerically
solving differential equations arising in engineering and
mathematical modeling.}

\item {} 
\sphinxAtStartPar
\sphinxhref{https://en.wikipedia.org/wiki/Solid\_mechanics}{Solid Mechanics} \sphinxhyphen{}
\sphinxstyleemphasis{Solid mechanics, also known as mechanics of solids, is the branch of
continuum mechanics that studies the behavior of solid materials,
especially their motion and deformation under the action of forces,
temperature changes, phase changes, and other external or internal
agents.}

\item {} 
\sphinxAtStartPar
\sphinxhref{https://en.wikipedia.org/wiki/Partial\_differential\_equation}{PDE}
\sphinxhyphen{} \sphinxstyleemphasis{In mathematics, a partial differential equation (PDE) is an
equation which imposes relations between the various partial
derivatives of a multivariable function.}

\end{itemize}


\bigskip\hrule\bigskip



\chapter{Changelog}
\label{\detokenize{index:changelog}}
\sphinxAtStartPar
The changelog is available
\sphinxhref{https://github.com/easycae-3d/myfempy/wiki}{here}


\bigskip\hrule\bigskip



\chapter{Project tree structure}
\label{\detokenize{index:project-tree-structure}}
\begin{sphinxVerbatim}[commandchars=\\\{\}]
/myfempy
\PYG{p}{|}\PYGZhy{}\PYGZhy{}/bin
\PYG{p}{|}   gui.py
\PYG{p}{|}   plotter.py
\PYG{p}{|}
\PYG{p}{|}
\PYG{p}{|}
\PYG{p}{|}\PYGZhy{}\PYGZhy{}/felib
\PYG{p}{|}   \PYG{p}{|}\PYGZhy{}\PYGZhy{}/fluid
\PYG{p}{|}   \PYG{p}{|}
\PYG{p}{|}   \PYG{p}{|}
\PYG{p}{|}   \PYG{p}{|}
\PYG{p}{|}   \PYG{p}{|}\PYGZhy{}\PYGZhy{}/fsi
\PYG{p}{|}   \PYG{p}{|}
\PYG{p}{|}   \PYG{p}{|}
\PYG{p}{|}   \PYG{p}{|}
\PYG{p}{|}   \PYG{p}{|}\PYGZhy{}\PYGZhy{}/struct
\PYG{p}{|}       beam21.py
\PYG{p}{|}       frame22.py
\PYG{p}{|}       frame23.py
\PYG{p}{|}       plane32.py
\PYG{p}{|}       plane42.py
\PYG{p}{|}       solid83.py
\PYG{p}{|}       spring20.py
\PYG{p}{|}       truss22.py
\PYG{p}{|}       .py
\PYG{p}{|}       .py
\PYG{p}{|}       .py
\PYG{p}{|}
\PYG{p}{|}   integrat.py
\PYG{p}{|}   material.py
\PYG{p}{|}   postproc.py
\PYG{p}{|}
\PYG{p}{|}
\PYG{p}{|}
\PYG{p}{|}\PYGZhy{}\PYGZhy{}/help
\PYG{p}{|}   help.py
\PYG{p}{|}   version.py
\PYG{p}{|}
\PYG{p}{|}
\PYG{p}{|}
\PYG{p}{|}\PYGZhy{}\PYGZhy{}/io
\PYG{p}{|}   filters.py
\PYG{p}{|}   ioctrl.py
\PYG{p}{|}   miscel.py
\PYG{p}{|}
\PYG{p}{|}
\PYG{p}{|}
\PYG{p}{|}\PYGZhy{}\PYGZhy{}/mesh
\PYG{p}{|}   meshgen.py
\PYG{p}{|}
\PYG{p}{|}
\PYG{p}{|}
\PYG{p}{|}\PYGZhy{}\PYGZhy{}/solver
\PYG{p}{|}       assembly.py
\PYG{p}{|}       bcloads.py
\PYG{p}{|}       plotter.py
\PYG{p}{|}       solverset.py
\PYG{p}{|}       static.py
\PYG{p}{|}       vibra.py
\PYG{p}{|}
\end{sphinxVerbatim}

\sphinxstepscope


\section{Basic Tutorial}
\label{\detokenize{tutorial:basic-tutorial}}\label{\detokenize{tutorial::doc}}
\sphinxstepscope


\section{Documentation}
\label{\detokenize{documentation:documentation}}\label{\detokenize{documentation::doc}}
\sphinxstepscope


\subsection{Introduction}
\label{\detokenize{introduction:introduction}}\label{\detokenize{introduction::doc}}
\sphinxstepscope


\subsection{Installation}
\label{\detokenize{installation:installation}}\label{\detokenize{installation::doc}}
\sphinxstepscope


\subsection{Examples}
\label{\detokenize{examples:examples}}\label{\detokenize{examples::doc}}


\renewcommand{\indexname}{Index}
\printindex
\end{document}